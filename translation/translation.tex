\documentclass{article}

% format
\usepackage[a4paper, total={6in, 9in}]{geometry}

% math symbols
\usepackage{amsmath}
\usepackage{amsthm}
\usepackage{amssymb}
\usepackage[shortlabels]{enumitem}
\usepackage{mathtools}
\usepackage{todonotes}

% theorems
\theoremstyle{definition}
\newtheorem{thm}{Theorem}
\newtheorem{prop}[thm]{Proposition}
\newtheorem{lemma}[thm]{Lemma}
\newtheorem{rmk}[thm]{Remark}
\newtheorem{defn}[thm]{Definition}
\newtheorem{cor}[thm]{Corollary}
\newtheorem{exo}[thm]{Exercise}

% definition equal
\newcommand{\defeq}{\vcentcolon=}
\newcommand{\eqdef}{=\vcentcolon}

% bibliography
\usepackage[backend=biber,style=alphabetic, sorting=ynt]{biblatex}

% url highlight
\usepackage{hyperref}

% algos
\usepackage[linesnumbered,ruled,vlined]{algorithm2e}
\SetArgSty{textnormal}

% bullet point style
\renewcommand{\labelitemi}{\tiny$\blacksquare$}

% new commands
\DeclareMathOperator{\dom}{dom}
\DeclareMathOperator{\Hess}{\textbf{H}}
\DeclareMathOperator{\Diag}{Diag}
\DeclareMathOperator{\Tr}{Tr}


\begin{document}
    \section{Upper bounds for the chromatic number
    of chordal graphs and their complements}

    \subsection{Discussion of the problem}

    A graph G such that
    $V(G)=\{v_1,\ldots, v_n\}$ is called 
    a chord intersection graph
    (or simply a chord graph) 
    if for some family $H = \{h_{i}, \ldots, h_{n}\}$
    of chords of a circle the 
    vertices $v_{i}$ and $v_{j}$
    are adjacent if and only if 
    $h_{i}$ and $h_{j}$ 
    intersect (i.e., have a common point
    not belonging to the circle). 
    Chord intersection graphs arise 
    when considering various
    combinatorial problems,
    from sorting problems to 
    the study of planar graphs
    or chain fractions [4, 24].
    In particular, the problem 
    of determining the smallest
    number of stacks necessary to
    realize a given permutation 
    of numbers reduces to finding 
    the chromatic number of 
    the corresponding chord graph [4, 16].

    A graph $G$ with $V(G) = \{v_{1}, \ldots, v_{n}\}$
    is called a meshing graph if 
    for some family $F = \{h_1, \ldots, h_n\}$ 
    of line segments the vertices 
    $v_{i}$ and $v_{j}$ are adjacent 
    if and only if the segments 
    $h_{i}$ and $h_{j}$ are meshing, 
    i.e., intersecting and $h_{i} \not \subseteq  h_{j}$
    and $h_{j} \not \subseteq h_{i}$.  
    Using stereographic projection, 
    it is easy to verify that $G$ is a chordal graph
    if and only if it is a meshing graph [4].

    The density and non-density
    of a chord graph can be found 
    in polynomial time of 
    the number of its vertices [4, 25]. 
    But no such algorithms are known 
    for finding the clique 
    number (i.e., chromatic 
    complement number)
    of chord graphs, and the 
    problem of finding the 
    chromatic number of chord
    graphs is NP-hard [26]. 
    It is all the more 
    important and interesting
    to find $\varphi(\mathcal{X}, k)$ and 
    $\varphi(\overline{\mathcal{X}}, k)$
    defined in the introduction, 
    where $\mathcal{X}$ is the 
    class of chord graphs and
    $\overline{\mathcal{X}}$
    is the class of their 
    complements. 
    The following results 
    are known in this direction. 
    Karapetyan [27] proved that 
    $4 \leq \varphi\left(\mathcal{X}, 2\right) \leq 8$
    $\varphi\left(\mathcal{X}, k\right) \leq k(k+1)/2$. 
    The second of these 
    results was later published in [28]. 
    Dyarfash [29] proposed a proof 
    of the statement $\varphi\left(\mathcal{X}, k\right) \leq 2^k k^2 (k-1)$. 
    Unfortunately, lemma 2 of [29] is incorrect. 
    Nevertheless, an estimate follows from [29] 
    $\varphi\left(\mathcal{X}, k\right) \leq 2^k (2^k - 2)k^2$. 
    Below we give an asymptotically exact 
    estimate for $k \rightarrow +\infty$
    of $\varphi\left(\mathcal{X}, k\right)$. 
    It turns out that $\varphi\left(\mathcal{X}, k\right) \sim k \log (k)$. 
    Furthermore, using the idea of 
    Dyarfash's proof [29], 
    we will show that $\varphi\left(\mathcal{X}, k\right) \leq 2^k(k+2)k$, 
    but not up to $\varphi\left(\mathcal{X}, 2\right) \leq 5$.  
    Note that the latter result 
    is announced but 
    not proved in [28, 30]. 
    In Section 4.4 we will show 
    $\varphi(\mathcal{X}, k) \geq (k/2)(\ln k - 2)$. 
    These results show that there 
    are graphs in $\mathcal{X}$
    that are more “complicated” 
    than graphs from the
    $\mathcal{D}$-class of all 
    intersection graphs of circle arcs. 
    It is known [4, 9, 31] that $\varphi\left(\mathcal{D}, k\right) = [3k/2]$, 
    $\varphi(\bar{\mathcal{D}}, k) = k+1$
    for $k \geq 1$.

    It is not difficult 
    to verify [4] that for 
    every family $H$ of chords of 
    a circle there is a family $H'$ 
    whose chord intersection graph 
    coincides with a similar graph 
    for $H$, but whose different 
    chords do not have common ends. 
    Therefore, in the proof 
    of the upperbound, we consider 
    only families of chords or intervals 
    whose different elements 
    do not have common ends.
    In order not to clutter the notation, 
    for a family $H$ of chords (intervals)
    we will denote by $\alpha\left(H\right)$, 
    $\chi\left(H\right)$, etc. 
    the non-density, chromatic number, etc. 
    for the corresponding chord 
    intersection graph (meshing graph). 
    Only finite families of 
    chords or segments are 
    considered throughout. 
    Let us condition some more notation. 
    The chord connecting points $a$ and $b$ 
    will be denoted as 
    an open segment $]a, b[$. 
    If $a$ and $b$ are 
    the points of the circle, 
    we denote by $[a, b]$, the arc connecting $a$ and $b$, 
    that when we move along it from $a$ to $b$ 
    we go around the circle counterclockwise. 
    We will write $]c, d[ \subseteq [a, b]$ 
    if both ends of the chord $]c, d[$ 
    belong to the arcs $[a, b]$. 
    In particular $]a, b[ \subseteq [a, b]$.

    \subsection{Upperbound for $\varphi\left(\bar{\mathcal{X}}, k\right)$.}

    \begin{thm}
        Let $H$ be a family 
        of chords of the 
        circle and $\alpha\left(H\right) = k $.
        Then $\sigma(H) = \chi(\bar{H}) \leq 
        \Psi(k) = \sum_{i = 1}^{[k/2]}[(k+1)/i] 
        + \varepsilon\left(k\right) - \left|Q(k)\right|$ 
        with $Q(k) = \{m \in Z | 1< (k+1)/4 < m < (k+1)/3\}$ 
        and $\varepsilon\left(k\right) = 0$ 
        if $k$ is even or 1 otherwise.
    \end{thm}
    \begin{proof}
        Let us call a chord 
        $h_0 = ]a, b[ \in H$ 
        a separating chord in $H$ 
        if there are such chords 
        $h_1$, $h_2$ such that 
        $h_1 \subseteq [a, b]$, 
        $h_2 \subseteq [b,a]$. 
        Let $H_1$ be a family 
        of non-separating chords 
        in $H$ and for all 
        $i = 2, 3, \ldots, |H|$, 
        let $H_i$ be a family of 
        non-separating chords in 
        $H \setminus \bigcup_{1 \leq j \leq i-1} H_{j}$.

        By definition of $H_i$, 
        if $h = ]a, b[ \in H_i$, $i > 1$, 
        then there exist chords 
        $h', h'' \in H_{i-1}$ 
        such that $h’ \subseteq [a, b]$, 
        $h' \subseteq [b, a]$. 
        The chord-connecting points 
        $a$ and $b$, can be written 
        $]a, b[$ or $]b, a[$. 
        In the following we denote 
        such a chord belonging to $H$ 
        by $]a, b[$ if the arc $[a, b]$ 
        does not contain chords of 
        $\bigcup_{j \geq i} H_j \setminus \{]a, b[\}$ 
        and by $]b, a[$ otherwise. 
        With this notation, 
        the graph of intersections 
        of chords of family 
        $H_i = \{]a^i_j, b^i_j[\}_{1 \leq j \leq m_i}$ 
        coincides with the graph 
        of intersections of arcs 
        of family $D_i = \{[a^i_j, b^i_j]\}_{1 \leq j \leq m_i}$. 
        But it has been noted above that, 
        according to [31],  we have $\sigma\left(H_{i}\right)= 
        \sigma(D_i) \leq \alpha(D_i)+1 = 
        \alpha(H_i) + 1$. 

        Let us show that $\alpha(H_i) \leq k/i$. 
        For $i = 1$ this follows from 
        the assumptions of the theorem. 
        Let $i > 1$. Let us choose $t = \alpha(H_i)$ 
        non-intersecting chords $]a^i_1, b^i_1[,
        \ldots ]a^t_i, b^t_i[$.
        By definition of $H_{i}$, for each
        $1 \leq j \leq t$, there exists
        $]a^{i-1}_{j}, b^{i-1}_{j}[ \in H_{i-1}$,
        such that $]a^{i-1}_{j}, b^{i-1}_{j}[
        \subseteq ]a^{i}_{j}, b^{i}_{j}[$.
        Similarly, if $i - 1> 1$, 
        then for each $1 \leq j \leq k$, 
        there exist chords $]a^{i-2}_{j}, b^{i-2}_{j}[
        \subseteq H_{i-2}$ such that
        $]a^{i-2}_{j}, b^{i-2}_{j}[ \subseteq 
        ]a^{i-1}_{j}, a^{i-1}_{j}[$ etc.
        
        Hence, there exists a
        family $H' = \left\{]a^{l}_{j}, b^{l}_{j}[
        \middle| l = 1, \ldots, i \text{ and }
        j = 1, \ldots, t\right\}$ of
        $\bigcup_{l = 1}^{i} H_{l}$
        of pairwise non-intersecting
        chords such that $\alpha\left(H_{i}\right) \leq k/i$.

        Assume that $H_{l} \neq \emptyset$.
        Let us consider $]a^{l}, b^{l}[
        \in H_{l}$. If $l > 1$,
        then there are chords
        $]a^{l-1}_{1}, b^{l-1}_{1}[ \in H_{l-1}$
        and $]a^{l-1}_{2}, b^{l-1}_{2}[ \in H_{l-1}$ 
        such that $]a^{l-1}_{1}, b^{l-1}_{1}[
        \in [a^{l}, b^{l}]$ and
        $]a^{l-1}_{2}, b^{l-1}_{2}[ \in [b^{l}, a^{l}]$.
        As in the previous argument,
        we assume the existence of a
        subset $H''$ of $H$ of
        pairwise non-intersecting chords
        with $H'' = \left\{]a^{l}, b^{l}[\right\}
        \cup \left\{]a^{i}_{j}, b^{i}_{j}
        \middle| i = 1, \ldots, l-1
        \text{ and } j = 1, 2[\right\}$.
        Thus $k \geq \left|H''\right|
        = 1 + 2\left(l-1\right)$
        and $l \leq (k+1)/2$.
        Hence $H_{i} \neq \emptyset$ 
        for all $i > (k+1)/2$
        and 
    \end{proof}

\end{document}

