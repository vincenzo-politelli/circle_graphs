\documentclass[12pt]{article}

% format
% \usepackage[a4paper, total={6in, 9in}]{geometry}                         

% math symbols
\usepackage{amsmath}
\usepackage{amsthm}
\usepackage{amssymb}
\usepackage[shortlabels]{enumitem} 
\usepackage{mathtools}
\usepackage{bbm}

% annotations
\setlength{\marginparwidth }{2cm}
\usepackage{todonotes}
\usepackage{pdfcomment}

% figures
\usepackage{graphicx}
\usepackage{subcaption}

% theorems
\theoremstyle{definition}
\newtheorem{thm}{Theorem}
\newtheorem{prop}[thm]{Proposition}
\newtheorem{lemma}[thm]{Lemma}
\newtheorem{rmk}[thm]{Remark}
\newtheorem{defn}[thm]{Definition}
\newtheorem{cor}[thm]{Corollary}
\newtheorem{exo}[thm]{Exercise}
\newtheorem{fact}[thm]{Fact}

% definition equal
\newcommand{\defeq}{\vcentcolon=}
\newcommand{\eqdef}{=\vcentcolon}
\pdfcommentsetup{color=yellow, opacity=0.5}

% bibliography
\usepackage[backend=biber,style=alphabetic, sorting=ynt]{biblatex}

% url highlight
\usepackage{hyperref}

% algos
\usepackage[linesnumbered,ruled,vlined]{algorithm2e}
\SetArgSty{textnormal}

% bullet point style
\renewcommand{\labelitemi}{\tiny$\blacksquare$}

% new commands
\DeclareMathOperator{\dom}{dom}
\DeclareMathOperator{\Hess}{\textbf{H}}
\DeclareMathOperator{\Diag}{Diag}
\DeclareMathOperator{\Tr}{Tr}
\DeclareMathOperator{\ind}{i}
\DeclareMathOperator{\sgn}{sign}

\begin{document}

    \section{Coloring circle graphs}
    
    \textbf{WARNING}: intervals
    $h_1$, $h_2$ overlap if
    $h_1 \cap h_2 \neq \emptyset$,
    $h_1 \not \subseteq h_2$ and
    $h_2 \not \subseteq h_1$.
    \vspace{4pt}

    We look at the class
    of circle graphs as the class
    of overlap graphs of intervals on a line.
    Without loss of generality,
    we only consider the interval
    models containing open intervals
    in which no two intervals
    share an endpoint.

    For any such family of intervals
    $F$ and any point $c$ on a line,
    we set $F^{-}\left(c\right) \defeq
    \left\{\left(a, b\right) \in F
    \middle| b < c\right\}$,
    $F^{0}\left(c\right) \defeq
    \left\{\left(a, b\right)
    \middle| a < c < b\right\}$,
    $F^{+}\left(c\right) \defeq
    \left\{\left(a, b\right)
    \middle| c < a\right\}$.
    If $\omega\left(F\right) = 2$,
    for all $\left(a, b\right) \in F$,
    we have that
    $F^{0}\left(a\right) \setminus F^{0}\left(b\right)$ 
    and $F^{0}\left(b\right) \setminus F^{0}\left(a\right)$
    only contain nested intervals.
    Otherwise, if we had two
    overlapping intervals $h_1, h_2 \in
    F^{0}\left(a\right) \setminus F^{0}\left(b\right)$,
    we would have $h_1, h_2, \left(a, b\right)$ 
    forming a triangle.

    \begin{lemma} \label{lemma:4.4}
        Let $F$ be the interval
        model of a circle graph.
        Suppose that $\omega\left(F\right) = 2$ 
        and that for no two
        $h_1, h_2 \in F$,
        there exists $h_3 \in F$ 
        with $h_3 \in h_1 \cap h_2$.
        Then, there exists a 3-coloring
        of $F$ such that 
        for all
        $\left(a, b\right) \in F$, 
        the intervals in $F^{0}\left(b\right) 
        \setminus F^{0}\left(a\right)$
        have the same color.
    \end{lemma}
    \begin{proof}
        By way of contradiction,
        suppose that the statement is incorrect
        and let $F = \left\{h_{i}\right\}_{i = 1}^{n} =
        \left\{\left(a_{i}, b_{i}\right)\right\}_{i = 1}^{n}$
        be the counterexample
        to this lemma with the
        least cardinality.
        Clearly, $G\left(F\right)$ is 
        connected.
        Let $\left(a_1, b_1\right),
        \ldots, \left(a_{t}, b_{t}\right)$,
        with $a_1 < \ldots < a_{t}$,
        be the intervals of $F$ which
        are not contained in any other 
        intervals. Notice that we have
        $F^{0}\left(a_{1}\right) = 
        F^{-}\left(a_1\right) = \emptyset$.
        Suppose that $\left|F^{0}\left(b_1\right)\right| \leq 1$.
        By the minimality of $F$,
        we get that the interval
        model $F' \defeq F \setminus \left\{h_{1}\right\}$ 
        is 3-colorable. It is clear
        that coloring $h_1$ with a different
        color from $h_2$, we obtain a coloring
        for $F$. Which contradicts the fact
        that $F$ is a counterexample to this lemma.
        Therefore, we conclude that $\left|F^{0}\left(b_1\right)\right| \geq 2$ 
        and $t \geq 2$.

        By the connectedness of $G\left(F\right)$,
        we have $h_{i} \cap h_{i+1} \neq \emptyset$ 
        for all $i \in \left\{1, \ldots, t -1\right\}$.
        Let us show that 
        for all
        $i \in \left\{1, \ldots, t - 1\right\}$
        we have
        $F^{-}\left(b_{i}\right) = \emptyset$ 
        or $F^{+}\left(a_{i+1}\right) = \emptyset$.
        Suppose that this does not
        hold for some $1 \leq i \leq t - 1$.
        Let $F_1 = F \setminus F^{+}\left(a_{i+1}\right)$
        and $F_2 = F \setminus F^{-}\left(b_{i}\right)$.
        
        We claim that $F_1 \cap F_2 = 
        \left\{h_{i}, h_{i+1}\right\}$.
        Indeed, $h_{j} \in 
        F_1 \cap F_2$, 
        iff $a_{j} \leq a_{i+1}$ 
        and $b_{j} \geq b_{i+1}$.
        Clearly $h_{i}$ and
        $h_{i+1}$ have this
        property.
        Now, suppose that there
        exists $h_{j} \in F_1 \cap F_2$
        such that $a_{j} < a_{i+1}$
        and $b_{j} > b_{i+1}$.
        We cannot have $a_{j} < a_{i}$,
        because otherwise $h_{i} \subseteq h_{j}$.
        Similarly, we cannot have $b_{j} > b_{i+1}$,
        because otherwise, we would
        have $h_{i+1} \subseteq h_{j}$.
        Therefore we have $a_{i} < a_{j}$ 
        and $b_{j} < b_{i+1}$.
        The only remaining case is
        that of $a_{i} < a_{j} < a_{i+1}$ 
        and $b_{i} < b_{j} < b_{i+1}$.
        But if this is the case,
        $\left\{h_{i}, h_{j}, h_{i+1}\right\}$ would
        form a triangle, which
        contradicts our assumptions.
        
        In view of the minimality of
        $F$, there exist 3-colorings
        $f_1$, $f_2$ of $F_1$ and $F_2$ as required
        by the statement of this lemma
        with colors 1, 2 and 3 (here
        we use our assumption that
        $F^{-}\left(b_{i}\right)$ 
        and $F^{+}\left(a_{i+1}\right)$ 
        are non-empty and thus
        $F_{1}$ and $F_{2}$ are
        strictly smaller than $F$).
        Since $h_1$ and $h_2$ overlap,
        they have different colors
        both in $f_1$ and $f_2$, therefore,
        we can assume without loss of 
        generality that
        $f_1\left(h_{i}\right) = f_2\left(h_{i}\right) = 1$
        and $f_{1}\left(h_{i+1}\right) = 
        f_2\left(h_{i+1}\right) = 2$.
        Let
        \begin{equation*}
            f\left(h\right) = 
            \begin{cases}
                f_1\left(h\right) &\text{ if } h \in F_1, \\
                f_2\left(h\right) &\text{ if } h \in F_{2}.
            \end{cases}
        \end{equation*}
        We verify that $f$ 
        is a coloring of $F$. Let
        $h' = \left(a', b'\right) \in F$,
        $h'' = \left(a'', b''\right) \in F$
        with $a' < a'' < b' < b''$.
        If $\left\{h', h''\right\} \subseteq F_1$ 
        or $\left\{h', h''\right\} \subseteq F_2$,
        then $f\left(h'\right) \neq 
        f\left(h''\right)$.
        If $h' \in F^{-}\left(b_{i}\right)$ 
        and $h'' \in F^{+}\left(a_{i+1}\right)$
        then, by assumption
        we have
        $h', h'' \not \subseteq \left(a_{i+1}, b_{i}\right)$,
        and so we must have $a' < a_{i+1} < 
        a'' < b' < b_{i} < b''$. 
        Since $f_2$ is a coloring,
        we have $f\left(h''\right) =
        f_2\left(h''\right) =
        f_2\left(h_{i+1}\right) = 2$.
        Since $h'$ overlaps with $h_{i+1}$
        and $h', h_{i+1}
        \in F_1$, then $f\left(h'\right)
        \neq 2 = f\left(h''\right)$.
        Similarly, 
        we can verify
        that for all $\left(a, b\right) \in F$
        all intervals in $F\left(b\right)
        \setminus F\left(a\right)$
        have the same color.
        Thus
        $f$ is a 3-coloring of $F$,
        which contradicts our choice of $F$.
        Thus, we have that
        $F^{-}\left(b_{i}\right) = \emptyset$ and
        $F^{+}\left(a_{i+1}\right) = \emptyset$
        for all $i \in \left\{1, \ldots, t-1\right\}$.

        We have $\left|F^{+}\left(a_{2}\right)\right| \geq
        \left|F^{0}\left(b_1\right)\right| \geq 2$,
        and, since $F^{-}\left(b_1\right) = \emptyset$,
        we have $F^{0}\left(a_2\right)
        = \left\{h_1\right\}$.
        
        If $F^{0}\left(b_2\right) = \emptyset$,
        i.e. $t=2$, then $h_2$
        only overlaps with $h_1$.
        A coloring $h_2$ with the color
        of the elements of $F^{0}\left(b_1\right)
        \setminus \left\{h_2\right\}$,
        we obtain a 3-coloring of $F$ with
        the conditions required by
        the present lemma, which is a
        contradiction.
        
        If $F^{0}\left(b_2\right) \neq \emptyset$
        and $t \geq 3$,
        we obtain (since $h_1 \in
        F^{-}\left(b_2\right)$)
        theat $F^{+}\left(a_3\right)=\emptyset$ 
        and thus $t=3$
        and $F^{0}\left(b_2\right) = \left\{h_3\right\}$.
        Consider $F \setminus \left\{h_2\right\}$.
        Since $F$ is minimal,
        we find that there exists
        a coloring $f$ of $F \setminus \left\{h_2\right\}$
        which follows the requirement
        of the present lemma.
        Let $h_2' = \left(a_2', b_2'\right)$
        be the longest interval of 
        $F^{0}\left(b_1\right) \setminus \left\{h_2\right\}$
        (which exists since $\left|F^{0}\left(b_1\right)\right|
        \geq 2$). This way
        $h_2'$ contains all
        the intervals in
        $F^{0}\left(b_1\right)$.
        Now, assume that $f\left(h_1\right) = 1$
        and $f\left(h_2'\right) = 2$.
        If $f\left(h_3\right)\neq 2$,
        then we have a coloring 
        of $F$ with the properties
        required by the present lemma.
        Which yields a contradiction.
        If $f\left(h_3\right) = 2$,
        then $h_3$ and $h_2'$
        do not overlap. Thus $h_3 \in F^{+}\left(b_2'\right)$.
        By assumption, all intervals
        of $F^{0}\left(b_2'\right)$ 
        are colored with the same color
        $\gamma \in \left\{1,3\right\}$.
        By recoloring all elements
        of $F^{+}\left(b_2'\right)$
        of color 2 to color 
        $\delta \in \left\{1,3\right\}
        \setminus \gamma$
        and all elements of $F^{+}\left(b_2'\right)$ 
        of color $\delta$ with color 2,
        we obtain a 
        new coloring $f'$ of 
        $F \setminus \left\{h_2\right\}$.
        Since $h_3 \in F^{+}\left(b_2'\right)$
        we get
        $f'\left(h_2'\right)=2$ and
        $f'\left(h_3\right) \neq 2$.
        By coloring $h_2$ with
        color 2, 
        we obtain a coloring
        of $F$ with the properties
        required by the present lemma.
    \end{proof}
    
    For each pair of overlapping
    intervals $h_1, h_2 \in F$,
    let $p\left(h_1, h_2\right)
    = h_1 \cap h_2$.
    We denote by $P\left(F\right)$ 
    the family of such intersections.
    Let $P^{0}\left(F\right) \subseteq 
    P\left(F\right)$ be the
    inclusion-wise maximal family
    of $P\left(F\right)$.

    \begin{lemma} \label{lemma:4.5}
        Let $F$ be a family of
        intervals with $\omega\left(F\right)=2$.
        Then, the intervals of $P^{0}\left(F\right)$ 
        do not intersect.
    \end{lemma}
    \begin{proof}
        Let $p_1=h_1 \cap h_2 \in 
        P^{0}\left(F\right)$ and
        $p_2=h_3 \cap h_4 \in P^{0}\left(F\right)$.
        Therefore, $\left(p_1 \cap p_2\right) =
        \left(a_4, b_1\right)$.
        So $a_1 < a_2 < a_4
        < b_1 <b_3 < b_4$.
        If $b_2 < b_4$,
        then $h_1, h_2, h_4$ 
        pairwise overlap. 
        Thus we have a triangle,
        which is a contradiction.
        Thus $b_4 < b_2$.
        By a symmetric argument
        we get $a_1 < a_3$.
        But then $p_2 =
        \left(a_4, b_3\right)
        \subsetneq \left(a_2, b_1\right)
        = p_1$ which contradicts
        the maximality of $p_2$.
    \end{proof}
    \begin{thm} \label{thm:circle}
        Let $F$ be a family of intervals
        with $\omega\left(F\right)=2$.
        $\chi\left(F\right) \leq 5$.         
    \end{thm}        
    \begin{proof}
        We construct the desired
        coloring by induction
        on $k$.
        Let $k = 1$.
        Let $F_1$ be the subset
        of intervals of $F$ that 
        do not lie in the intersection
        of overlapping intervals of $F$.
        By Lemma \ref{lemma:4.4},
        there exists a coloring
        $f_1$ of $F_1$ with colors
        $\left\{1,2,3\right\}$ such
        that for every $\left(a,b\right)
        \in F_1$, the intervals
        of the family $F^{0}_{1}
        \left(b\right) \setminus F^{0}\left(a\right)$
        are colored with the same
        color.
        By Lemma \ref{lemma:4.5},
        the intervals of the family
        $P^{0}\left(F_1\right)$ 
        do not overlap. By definition
        of $F_1$, each interval in 
        $F \setminus F_1$ is contained
        in some (maximal) intersection
        $p \in P^{0}\left(F_1\right)$.
        Consider that the following
        construction has been done
        for $k - 1$ with $k \geq 2$.
        We want the following properties
        to be true.
        \begin{enumerate}
            \item The intervals in
                $F' \defeq 
                \bigcup_{i=1}^{k-1} F_{i}$
                (where the $F_{i}$s
                might not be disjoint)
                are colored with colors
                $\left\{1, \ldots, 5\right\}$.
            \item The intervals in
                $F \setminus F'$ do not
                overlap with intervals
                in $\bigcup_{i=1}^{k-2}
                F_{i} \setminus F_{k-1}$.
                \label{cond:2}
            \item \label{cond:3} All intervals in 
                $F \setminus F'$ 
                are contained in some 
                interval $p \in 
                P^{0}\left(F^{k-1}\right)$.
            \item For each $p =
                \left(c, d\right) \in 
                P^{0}\left(F_{k-1}\right)$ 
                either only one
                color is used to color
                all intervals in 
                $F^{0}_{k-1}\left(c\right) 
                \setminus F^{0}_{k-1}\left(d\right)$,
                and no more than 
                two colors are used to 
                colors intervals
                of $F^{0}_{k-1}\left(d\right)
                \setminus F^{0}_{k-1}\left(c\right)$
                or viceversa.
        \end{enumerate}
        Let us show how to carry out the 
        $k$-th step of this construction.
        If $F \setminus F' = \emptyset$,
        then we are done and we have 
        the desired coloring.
        Consider any interval
        $p = \left(c, d\right)
        \in P^{0}\left(F_{k-1}\right)$,
        which contains at least one interval
        of $F \setminus F'$. 
        Let
        \begin{align*}
            F^{0}_{k-1}\left(c\right) \setminus 
            F^{0}_{k-1}\left(d\right)
            &= \left\{h_{j}\right\}_{j=1}^{t}
            = \left\{\left(c_{j},
            d_{j}\right)\right\}_{j=1}^{t} \\
            F^{0}_{k-1}\left(d\right) \setminus 
            F^{0}_{k-1} \left(c\right)
            &= \left\{h_{j}\right\}_{j=t+1}^{s}
            = \left\{\left(c_{j},
            d_{j}\right)\right\}_{j=t+1}^{s},
        \end{align*}
        and $p = \left(c_{s}, d_{s}\right)
        \cap \left(c_{t}, d_{t}\right) = 
        \left(c_{s}, d_{t}\right)$.
        As already noted, 
        each
        family consists of
        nested intervals.
        Without loss of generality,
        suppose that
        $\gamma_1 \in \left\{1, 3\right\}$
        is the color used to
        color the intervals
        $h_{t+1}, \ldots, h_{s}$
        and 
        let $\gamma_{2}, \gamma_{3}$
        be the colors used to 
        color
        $\left\{h_1, \ldots h_t\right\}$.
        
        Let $I_{p}$ be the
        set of intervals in $F$ 
        contained in $p$,
        the intervals $h_{t+1},
        \ldots, h_{s}$ and the 
        interval 
        $\left(d, d_{s} + 1\right)$.
        Let $F_{k, p}$ be 
        the set of intervals in $I_{p}$ 
        which are not contained in
        the overlap of any two
        intervals in $I_{p}$. 

        Let us show that the 
        intervals from $I_{p}
        \setminus F_{k, p}$ 
        do 
        not overlap with 
        any of the intervals of
        $F' \setminus \left\{
        h_{t+1},\ldots,h_{s}\right\}$.
        By condition \ref{cond:2}
        of our induction hypothesis, 
        intervals from
        $F' \setminus \left\{h_{t+1},
        \ldots, h_{s}\right\}$
        that can overlap with 
        elements in $I_{p} \setminus 
        F_{k,p}$ 
        are only intervals
        in $\left\{h_1, \ldots, h_{t}\right\}$.
        Let $\left(a_1, b_1\right) \in I_{p}$ 
        be an interval contained in the
        overlapping of
        $\left(a_2, b_2\right), 
        \left(a_3, b_3\right)
        \in I_{p}$ and thus
        overlapping with some
        $h_{j} =\left(c_{j}, d_{j}\right)$ 
        for $1 \leq j \leq t-1$.
        Then $a_{i} > c_{s}$ for
        all $i \in \left\{1, 2, 3\right\}$.
        We also get $c_{j} < c_{s}$,
        $a_1 < d_{j} < b_1$. Hence
        $h_{j}$, $\left(a_2, b_2\right)$ 
        and $\left(a_3, b_3\right)$ 
        pairwise overlap.
        Which is a contradiction.

        If $p$ contains at least one interval,
        then $F_{k, p} \setminus \left(h_{t+1},
        \ldots, h_{s}\right) \cup 
        \left\{\left(d, d_{s} + 1\right)\right\}
        \neq \emptyset$.
        By Lemma \ref{lemma:4.4}
        (replacing $F^{0}\left(b\right)
        \setminus F^{0}\left(a\right)$
        with $F^{0}\left(a\right)
        \setminus F^{0}\left(b\right)$),
        we can color $F_{k, p}$ 
        with colors from
        $\left\{1,2,3,4,5\right\} \setminus 
        \left\{\gamma_2, \gamma_3\right\}$
        such that
        for each
        $\left(a, b\right) \in 
        F_{k, p}$,
        the interval from
        $F^{0}_{k, p}\left(a\right) \setminus 
        F^{0}_{k, p}\left(b\right)$
        have the same color.
        At the same time,
        since $\left\{h_{t+1},
        \ldots, h_{s}\right\}
        = F^{0}_{k, p}\left(d\right)
        \setminus F^{0}_{k, p}\left(d_{s} + 1\right)$
        we can assume that 
        $h_{t+1}, \ldots,
        h_{s}$ is colored 
        with (only) $\gamma_1$.
        
        Let us denote $F_{k, p}' = 
        F_{k, p} \setminus \left\{
        \left(d, d_{s} + 1\right)\right\}$.
        It is easy to see that the coloring of
        $F_{k, p}'$ is compatible
        with that of
        $F'$.
        Carrying out similar constructions
        for each $p \in P^{0}\left(F_{k-1}\right)$
        containing at least one uncolored
        interval of $F$, let
        $F_{k} \defeq \bigcup_{p \in 
        P^{0}\left(F_{k-1}\right)} F_{k, p}'$.

        One can check that the 
        induction hypotheses hold
        for $k$.
        Also, the number of uncolored
        vertices strictly decreseas 
        at each step. Therefore
        the coloring will be 
        completed in a finite
        number of iterations.
    \end{proof}
    
    \section{Coloring polygon circle graphs}

    We define polygon circle
    graphs as the set of intersection
    graphs of polygons
    (including segments)
    inscribed in a circle.
    For the sake of simplicity,
    we will only consider
    families of polygons
    in which no two
    vertices coincide.
    The class of circle
    graphs can be obtained
    by restricting the
    choice of polygons
    to segments.

    Like circle graphs, we
    handle polygon
    circle graphs by looking
    at the stereographic projection
    of the circle onto $\mathbb{R}$.
    Thus, a polygon circle
    graph is represented by a 
    family $\left\{h_{i}\right\}_{i=1}^{n}$
    with 
    $\left(a_{1}^{\left(i\right)},
    \ldots, a_{k_{i}}^{\left(i\right)}\right)$ 
    for some $k_{i} \geq 1$ 
    with $a^{\left(i\right)}_{j} \in \mathbb{R}$
    being the image of
    the stereographic projection
    of a vertex of $h_{i}$.

    Given a familiy
    $F = \left\{h_{i}\right\}_{i=1}^{n}$ 
    of polygons, we denote
    the corresponding
    intersection graph $G\left(F\right)$.
    By abuse of notation,
    we will denote
    $\omega\left(G\left(F\right)\right)$ 
    by $\omega\left(F\right)$.
    Given two polygons
    $h_{i}, h_{i'}$ in $F$,
    we say that $h_{i}$ and
    $h_{i'}$ overlap 
    if there exists indices 
    $j$ and $j'$ such that
    $\left(a_{j}^{\left(i\right)},
     a_{j+1}^{\left(i\right)}\right)$ 
     and $\left(a_{j'}^{\left(i'\right)},
     a_{j'+1}^{\left(i'\right)}\right)$ 
     overlap.

     Given a polygon
     $h \defeq \left(
     a_{1}, \ldots, a_{k}\right)$,
     a contraction of $h$,
     is a polygon $h'$ 
     of the form
     $\left(a_{1}, \ldots,
     a_{i-1}, a_{i+1}, \ldots
     a_{k}\right)$ 
     where indices are
     unerstood modulo $k$.
     Consider a family of
     polygons
     $F = \left\{h_{i}\right\}_{i=1}^{n}$.
     Let $F'$ be the family
     of polygons generated
     by contracting one
     polygon in $F$.
     We define the partial order
     `$\prec$' on the set
     of polygon circle graphs
     to be the order generated
     by the relations $F' \prec F$.
     Notice that $\prec$ is
     well-founded since,
     if $F' \prec F$,
     the total number of 
     vertices of polygons in
     $F'$ is strictly less than
     that of the polygons in $F$.

     Let $c \in \mathbb{R}$.
     We establish the following notation.
     \begin{align*}
         F^{0}\left(c\right) &\defeq
         \left\{h_{i} \in F \;\middle|\;
         a_{1}^{\left(i\right)}
         < c < a_{k_{i}}^{\left(i\right)}\right\}, \\
         F^{+}\left(c\right) &\defeq
         \left\{h_{i} \in F \middle|
         c < a_{1}^{\left(i\right)}\right\},\\
         F^{-}\left(c\right) &\defeq
         \left\{h_{i} \in F
         \middle| a_{k_{i}}^{\left(i\right)} < c\right\}.
     \end{align*}
     We also importantly define for
     $1 \leq i \leq n$ and 
     $1 < j \leq k_{i}$, 
     the following:
     \begin{gather*}
         \widetilde{F}^{0}\left(a_{j}^{\left(i\right)}\right)
         \defeq \left\{
         h_{i'} \in F \;\middle|\;
         a_{1}^{\left(i\right)} < a_{j'}^{\left(i'\right)}
         < a_{j}^{\left(i\right)}
         < a_{j'+1}^{\left(i'\right)}
         \text{ for some }
         1 \leq j' < k_{i'} \right\},
     \end{gather*}
     For any $h_{i} \in F^{0}\left(c\right)$,
     denote $I_{c}\left(h_{i}\right)
     \defeq \left(a_{j}^{\left(i\right)},
     a_{j+1}^{\left(i\right)}\right)$ 
     such that $c \in 
     \left(a_{j}^{\left(i\right)},
     a_{j+1}^{\left(i\right)}\right)$ 
     and $O_{c}\left(h_{i}\right)
     \defeq \left(a_{1}^{\left(i\right)},
     a_{k_{i}}^{\left(i\right)}\right)$.
     For any polygon $h_{i}$,
     we call the segment
     $\left(a_{1}^{\left(i\right)},
     a_{k_{i}}^{\left(i\right)}\right)$
     the external
     segment of $h_{i}$.

     Also, for any two polygons
     $h_{i}$, $h_{i'}$ be two
     polygons whose external segments
     overlap with
     $a_1^{\left(i\right)} < a_1^{\left(i'\right)}$.
     Let $\left(a_{j}^{\left(i\right)},
     a_{j+1}^{\left(i\right)}\right) =
     I_{a_1^{\left(i'\right)}}\left(h_{i}\right)$
     and let
     $b^{\left(i, i'\right)} \defeq a_{j+1}^{\left(i\right)}$.
     Let $\left(a_{j'}^{i'}, a_{j'+1}^{\left(i'\right)}\right)
     = I_{b^{\left(i,i'\right)}}\left(h_{i'}\right)$ 
     and let $a^{\left(i',i\right)}
     \defeq a_{j'}^{\left(i'\right)}$.
     We say that the interval
     $\left(a^{\left(i', i\right)},
     b^{\left(i, i'\right)}\right)$ 
     is the overlap of $h_{i}$, $h_{i'}$.

     Consider the set of intervals
     of the form $\left(a^{\left(i',i\right)},
     b^{\left(i,i'\right)}\right)$ 
     and denote it $P\left(F\right)$.
     Take the inclusion-wise maximal
     subfamily of such intervals
     and denote it $P^{0}\left(F\right)
     \subseteq P\left(F\right)$.
     
     \begin{rmk}
         Notice that, if 
         $\omega\left(F\right) = 2$,
         the maximality of
         $P^{0}\left(F\right)$ 
         implies that all the
         intervals in $P^{0}\left(F\right)$
         are disjoint.
     \end{rmk}

     From now on, consider
     $F = \left\{h_{i}\right\}_{i = 1}^{n}$ 
     be a family of polygons
     with $\omega\left(F\right)=2$.
     We have the following two facts.

     \begin{fact}
         For any $\left(a_{j}^{\left(i\right)}\right)$,
         the elements of
         $\widetilde{F}^{0}\left(a_{j}^{\left(i\right)}\right)$ 
         are nested. That is, let
         \begin{gather*}
             F^{0}\left(a_{j}^{\left(i\right)}\right) \setminus 
             \left(F^{0}\left(a_{1}^{\left(i\right)}\right) \cap 
             F^{0}\left(a_{k_{i}}^{\left(i\right)}\right)\right)
             \eqdef \left\{h_1, \ldots, h_{l}\right\}.
         \end{gather*}
         Then, up to relabeling
         the elements of $F$, we have that
         \begin{gather*}
             I_{a_{j}^{\left(i\right)}}\left(h_1\right)
             \subseteq I_{a_{j}^{\left(i\right)}}\left(h_2\right)
             \subseteq \cdots
             \subseteq I_{a_{j}^{\left(i\right)}}\left(h_{l}\right), \\
             O_{a_{j}^{\left(i\right)}}\left(h_1\right)
             \subseteq O_{a_{j}^{\left(i\right)}}\left(h_2\right)
             \subseteq \cdots
             \subseteq O_{a_{j}^{\left(i\right)}}\left(h_{l}\right).
         \end{gather*}
     \end{fact}
     
     \begin{fact}
         For any two polygons
         $h \in \widetilde{F}^{0}\left(a_{j}^{\left(i\right)}\right)$ 
         and 
         $h' \in \widetilde{F}^{0}\left(a_{j'}^{\left(i\right)}\right)$, 
         we have that $h$ and $h'$
         do not overlap.    
     \end{fact}

     We now prove the following
     technical lemma.

     \begin{lemma} \label{lemma:poly}
         Let $F \defeq \left\{h_{i}\right\}_{i =1}^{n}$ 
         be a family of polygons with
         $\omega\left(F\right) = 2$.
         Suppose that for any
         $\left(a, b\right) \in P^{0}\left(F\right)$,
         there is no $h_{i}$ such
         that $a < a_1^{\left(i\right)}
         < a_{k_{i}}^{\left(i\right)} < b$.
         Then $\chi\left(F\right) \leq 3$.
         Moreover, there exists
         a 3-coloring of $F$ such
         that for all $1 \leq i \leq n$
         and $1 < j \leq k_{i}$,
         all polygons in
         $\widetilde{F}^{0}\left(a_{j}^{\left(i\right)}\right)$.
         We call such coloring a
         `good coloring'.
     \end{lemma}

     \begin{proof}
         By way of contradiction,
         suppose that the statement is
         false and let
         $F = \left\{h_{i}\right\}_{i = 1}^{n}$ 
         be the smallest counterexample
         to this lemma with respect
         to the order $\prec$.
         Up to relabeling the elements
         of $F$, suppose without loss
         of generality that
         $h_{1}, \ldots, h_{t}$ are
         the polygons of $F$ whose
         external segments (seen as itervals
         of the form
         $\left(a_1^{\left(i\right)}, a_{k_{i}}^{\left(i\right)}\right)$) 
         are not contained
         in the external segment of 
         any other polygon of $F$.
         We further assume that
         $a_1^{\left(1\right)} <
         a_1^{\left(2\right)} < \ldots < 
         a_1^{\left(t\right)}$.
         By the minimality of $F$, we get
         that $G\left(F\right)$ is
         connected and $h_{i} \cap h_{i+1} \neq \emptyset$
         for all $i \in \left\{1, \ldots, t-1\right\}$.
         Therefore, we have that the points
         $a^{\left(i+1, i\right)}$ and
         $b^{\left(i, i+1\right)}$ are
         well-defined for all 
         $i \in \left\{1, \ldots, t-1\right\}$.

         First, suppose that $t = 1$.
         In this case, we have $F^{0}\left(a_1^{\left(1\right)}\right)
         = \emptyset$ and
         $F^{0}\left(a_{k_1}^{\left(1\right)}\right) = \emptyset$.
         Thus, contracting
         $a_{k_{i}}^{\left(1\right)}$ in $h_1$,
         either does not change the underlying graph,
         or implies that $G\left(F\right)$ is
         disconnected. In both cases, we have a contradiction.

         We are now going to prove that
         for all $i \in \left\{1, \ldots, t -1\right\}$,
         we have either $F^{-}\left(b^{\left(i, i+1\right)}\right)
         = \emptyset$ or 
         $F^{+}\left(a^{\left(i+1, i\right)}\right)
         = \emptyset$. This is
         not the case for $t \geq 4$ since,
         in such case, we would have 
         $h_1 \in F^{-}\left(b^{\left(2, 3\right)}\right)$
         and $h_{t} \in F^{+}\left(a^{\left(3, 2\right)}\right)$.
         Thus, this would imply $t \leq 3$.

         By way of contradiction,
         suppose that there exists
         $i_0 \in \left\{1, \ldots, t-1\right\}$
         such that $F^{-}\left(b^{\left(i_0, i_0+1\right)}\right)
         \neq \emptyset$ and
         $F^{+}\left(a^{\left(i_0+1, i_0\right)}\right)
         \neq \emptyset$.
         By the minimality of $F$, 
         we have that
         $F_1 \defeq F \setminus F^{+}
         \left(a^{\left\{i_0+1, i_0\right\}}\right)$
         and
         $F_2 \defeq F \setminus F^{-}
         \left(b^{i_0,i_0+1}\right)$ 
         admit good colorings
         $f_1$ and $f_2$.
         Notice that $F_1 \cap F_2 = 
         \left\{h_{i_0}, h_{i_0+1}\right\}$.
         So, we can assume without loss
         of generality that
         $f_1\left(h_{i_0}\right) = 
         f_2\left(h_{i_0}\right) = 1$ 
         and $f_1\left(h_{i_0+1}\right) =
         f_2\left(h_{i_0+1}\right) =2$.
         This also allows us to define 
         the function $f \vcentcolon 
         F \rightarrow \left\{1, 2, 3\right\}$ 
         with $f\left(h\right) = f_1\left(h\right)$ 
         if $h \in F_1$ and 
         $f\left(h\right) = f_2\left(h\right)$ 
         if $h \in F_2$.
         By the maximality
         of $h_{i_0}$ and $h_{i_0+1}$,
         a family
         $\widetilde{F}^{0}\left(a_{j}^{\left(i\right)}\right)$
         is either entirely contained
         into $F_1$ or $F_2$.
         Thus, if $f$ is a coloring,
         it is also a good coloring.

         We now prove that $f$ is a coloring.
         Let $h_{i}, h_{i'} \in F$ be overlapping
         polygons. If both $h_{i}, h_{i'} \in F_1$ 
         or $h_{i}, h_{i'} \in F_2$, then 
         $f\left(h_{i}\right) \neq f\left(h_{i'}\right)$
         by the fact that $f_1$ and $f_2$ 
         are colorings.
         Suppose that $h_{i} \in F_1$ and
         $h_{i'} \in F_2$, then, since for no
         $\left(a, b\right) \in P^{0}\left(F\right)$ 
         and $h_{i} \in F$, we deduce that
         \begin{gather*}
             a_1^{\left(i\right)} < 
             a^{\left(i_0+1, i_0\right)} <
             a_1^{\left(i'\right)} < 
             a_{k_{i}}^{\left(i\right)} < 
             b^{\left(i_0, i_0+1\right)} < 
             a_{k_{i'}}^{\left(i'\right)}.
         \end{gather*}
         Now, since $h_{i_0+1}, h_{i'} \in 
         \widetilde{F}^{0}\left(b^{\left(i_0+1, i_0\right)}\right)$,
         and $h_{i_0+1}, h_{i'} \in F_2$,
         since $f_2$ is a good
         coloring, we have that $f\left(h_{i'}\right) =
         f\left(h_{i_0+1}\right) = 2$.
         Now, since $h_{i_0+1}$ and $h_{i}$ 
         overlap, 
         $h_{i_0+1}, h_{i} \in F_1$,
         and $f_1$ is a coloring, we get
         that $f\left(h_{i_0+1}\right) \neq
         f\left(h_{i}\right)$.
         Therefore, $f\left(h_{i}\right) \neq 
         f\left(h_{i'}\right) = 2$.
         So $f$ is a coloring and so
         a good coloring.
         This contradicts the definition
         of $F$.

         We thus conclude that
         $F^{-}\left(b^{\left(i, i+1\right)}\right) = \emptyset$ 
         or $F^{+}\left(a^{\left(i+1, i\right)}\right) = \emptyset$
         for all
         $i \in \left\{1, \ldots, t-1\right\}$ and
         $t \leq 3$.

         Suppose now that $t = 2$.
         Suppose that there exists
         $b^{\left(1,2\right)} < 
         a_{j}^{\left(2\right)} <
         a_{k_2}^{\left(2\right)}$.
         Then, since $F^{0}\left(a_{k_2}^{\left(2\right)}\right) = 
         \emptyset$, we have that 
         contracting $a_{k_2}^{\left(2\right)}$
         in $h_2$, does not change the underlying 
         graph. Which is not possible by
         the definition of $F$.
         Therefore $a^{\left(2,1\right)} =
         a_{k_2}^{\left(2\right)}$. By a
         symmetric reasoning, we can show
         $b^{\left(1, 2\right)} = a_2^{\left(1\right)}$.
         
         Consider the case of
         $F^{+}\left(a^{\left(2, 1\right)}\right) = \emptyset$.
         Since $b^{\left(1, 2\right)} =
         a_{2}^{\left(2\right)}$, $h_1$ only
         overlaps with $h_2$.
         By the minimality of $F$, 
         we get a good coloring 
         $f'$ of $F \setminus \left\{h_1\right\}$.
         We can extend $f'$ into 
         a good coloring of $F$,
         by coloring $h_1$ with 
         a color different from
         $f'\left(h_2\right)$.

         Consider the case of
         $F^{-}\left(b^{\left(1, 2\right)}\right) = \emptyset$.
         Since $a^{\left(2, 1\right)} = 
         a_{k_2}^{\left(2\right)}$,
         $h_2$ only intersecs with $h_1$.
         Now, take the outermost polygons
         overlapping with $h_1$ and
         call them $h_{i_1}, \ldots, h_{i_{l}}$ with
         \begin{gather*}
             a_1^{\left(i_1\right)} <
             a_{k_{i_1}}^{i_1} <
             a_1^{\left(i_2\right)} < 
             \ldots <
             a_{k_{i_{l - 1}}}^{\left(i_{l - 1}\right)} <
             a_1^{\left(i_{l}\right)} <
             a_{k_{i_{l}}}^{i_{l}}.
         \end{gather*}
         We can thus partition all the
         polygons overlapping with
         $h_1$ by looking at which $h_{i_{j}}$ 
         the are nested in. Denote
         such class by $N_{h_{i_{j}}}$.
         Notice that we have a good coloring
         $f'$ of $F \setminus \left\{h_2\right\}$.
         We can extend $f'$ into a colring
         $f$ of $F$ by coloring $h_2$ with
         the same color as the other polygons in
         $F^{0}\left(a_2^{\left(1\right)}\right)$.

         For the sake of simplicity,
         suppose that we have
         $f\left(h_1\right) = 1$,
         $f\left(h_{i_1}\right) = 2$ and
         $f\left(h_2\right) = 2$.
         Notice that all the polygons
         in a class $N\left(h_{i_{j}}\right)$ 
         have the same color.

         In order to obtain a good coloring
         of $F$, it suffices to color all the polygons
         overlapping with $h_1$ with color 2.
         If one exists, take the smallest index
         $i_{k}$ ($\neq i_1$) such that 
         the color of $N\left(h_{i_{k}}\right)$ 
         is different from 2. So $f\left(h_{i_{k}}\right) = 3$.
         Define $O \defeq F^{+}\left(a_{k_{i_{k-1}}}^{i_{k-1}}\right)$.
         Notice that the polygons 
         in $F \setminus O$, that overlap
         with polygons in $O$ lie in
         $O' \defeq F^{0}\left(a_{k_{i_{k-1}}}^{i_{k-1}}\right) =
         \widetilde{F}^{0}\left(a_{k_{i_{k-1}}}^{\left(i_{k-1}\right)}\right)$
         by the maximality of $h_{i_{k-1}}$.
         All polygons in $O'$ are colored
         with color 1.
         We can thus swap the colors
         2 and 3 of the polygons 
         contained in $O$, so that
         the classes $N\left(h_{i_1}\right),
         \ldots, N\left(h_{i_{k}}\right)$ are
         colored with color 2.
         We can repeat this procedure until
         all classes $N\left(h_{i_{j}}\right)$ 
         are colored with color 2.

         We now tackle the case of $t = 3$.
         If $t = 3$, we have $h_3 \in F^{+}
         \left(a^{\left(2, 1\right)}\right)$ 
         and $h_1 \in F^{-}\left(b^{\left(2, 3\right)}\right)$.
         Thus, we have that $F^{+}\left(b^{\left(2, 1\right)}\right)
         \neq \emptyset$.
         By similar arguments as those
         used in the previous case,
         the minimality of $F$ implies
         $a^{\left(2, 1\right)} = a_1^{\left(2\right)}$,
         $b^{\left(2, 1\right)} = a_2^{\left(1\right)}$,
         $a^{\left(3, 2\right)} = a_{k_3 - 1}^{\left(3\right)}$ and
         $b^{\left(2, 3\right)} = a_{k_2}^{\left(2\right)}$.

         Also, suppose that there exists
         $a_2^{\left(1\right)} < a_{j}^{\left(2\right)} <
         a_{k_1}^{\left(1\right)}$ for
         some $j > 1$. Then,
         since $F^{-}\left(b^{\left(2, 1\right)}\right) = \emptyset$, 
         contracting $a_1^{\left(2\right)}$ 
         in $h_2$ does not change the
         underlying graph, which
         contradicts the definition of $F$.
         Similarly, there is no
         $j < k_2$ $a_1^{\left(3\right)} <
         a_{j}^{\left(2\right)} <
         a_{k_3-1}^{\left(3\right)}$.
         Therefore, for all
         $1 < j < k_2$, we have
         that $a_{k_1}^{\left(1\right)} < 
         a_{j}^{\left(2\right)} < 
         a_1^{\left(3\right)}$.

         Suppose that there exists
         $1 < j < k$ such that
         $a_{k_1}^{\left(1\right)} <
         a_{j}^{\left(2\right)} <
         a_{j}^{\left(3\right)}$.
         Contract $a_{k_2}^{\left(2\right)}$ 
         in $h_2$ (call the resulting
         polygon $h_2'$)
         and get a good coloring
         $f'$ of the resulting family
         of polygons.
         Suppose without loss of
         generality that $f\left(h_1\right) = 1$,
         $f'\left(h_2'\right) = 2$.
         If $f'\left(h_3\right) \neq 2$, 
         then we can decontract $h_2'$ 
         to get a good coloring of $F$, 
         which is a contradiction.
         Let us consider the case
         $f'\left(h_3\right) = 2$.
         Define $O' \defeq F^{+}\left(a_{k_2 - 1}^{\left(2\right)}\right)$.
         Notice that all the polygons in
         $F \setminus O'$ that intersect with
         $O'$ lie in the set 
         $F^{0}\left(a_{k_2-1}^{\left(2\right)}\right) =
         \widetilde{F}^{0}\left(a_{k_2-1}^{\left(2\right)}\right)$
         by the maximality of $h_2$.
         By the fact that $f'$ is a
         good coloring, all polygons in $F^{0}\left(
         a_{k_2-1}^{\left(2\right)}\right)$
         share the same color $\gamma \in \left\{1, 3\right\}$.
         Let $\delta \in \left\{1, 3\right\} 
         \setminus \left\{\gamma\right\}$
         We can swap the colors 2 and $\delta$
         in $O'$ by preserving the good coloring.
         We can finally decontract $h_2$ to get
         a good coloring of $F$. 
         A contradiction.

         Thus, we deduce that $h_2$ is a segment.
         We get a good coloring $f'$ of
         $F \setminus \left\{h_2\right\}$.
         Suppose without loss of generality
         that $f'\left(h_1\right) = 1$.
         We define $h_{i_1}, \ldots, h_{i_{l}}$ 
         analogously as in the previous case.
         By the same argument as in the previous case,
         we can color all the sets 
         $N\left(h_{i_{j}}\right)$ with the
         same color as $N\left(h_{i_1}\right)$,
         say 2 without loss of generality.
         If after this recoloring,
         $f'\left(h_3\right) \neq 2$, 
         we can extend $f'$ into a
         good coloring of $F$ by 
         coloring $h_2$ with color 2.
         This gives a contradiction.
         We thus have $f'\left(h_3\right) = 2$.
         Let $\gamma \in \left\{1, 3\right\}$
         be the color of the polygons
         in $F^{0}\left(a_{k_{i_{l}}}^{\left(i_{l}\right)}\right)
         = \widetilde{F}^{0}\left(
         a_{k_{i_{l}}}^{\left(i_{l}\right)}\right)$.
         Let $O'' \defeq F^{+}\left(
         a_{k_{i_{l}}}^{\left(i_{l}\right)}\right)$.
         Notice that all the elements
         in $F \setminus O''$ 
         that intersect elements in $O''$
         are contained in $F^{0}\left(
         a_{k_{i_{l}}}^{\left(i_{l}\right)}\right)$.
         Let $\delta \in \left\{1, 3\right\}
         \setminus \left\{\gamma\right\}$.
         Therefore, we can swap colors
         2 and $\delta$ in $O''$
         by preserving the good coloring property.
         We can thus extend $f'$ into a
         good coloring of $F$ by
         coloring $h_2$ with color 2
         to obtain a good coloring of $F$.
         Again, a contradiction.
     \end{proof}

     We first establish some more notation.
     Namely, using the same notational conventions
     used up to this point, we denote, for
     $1 \leq i' \leq n$ and $1 \leq j' < k_{i'}$

     \begin{gather*}
         \overline{F}^{0}\left(a_{j}^{\left(i\right)}\right) \defeq
         \left\{h_{i'} \in F \;\middle|\;
         a_{j}^{\left(i\right)} <
         a_{j'}^{\left(i'\right)} <
         a_{j+1}^{\left(i\right)} <
         a_{k_{i'}}^{\left(i'\right)}
         \text{ for some }
         1 \leq j < k_{i}\right\}
     \end{gather*}

     We remark in passing that, by symmetry,
     Lemma \ref{lemma:poly} remains 
     true if we
     replace $\widetilde{F}^{0}\left(a_{j}^{\left(i\right)}\right)$ 
     with $\overline{F}^{0}\left(a_{j}^{\left(i\right)}\right)$.
     This is because, the case of $t = 1$
     and the proof of $t \leq 3$ are exactly
     the same. Also, when 
     dealing with the cases $t = 2$ 
     and $t = 3$, we use the minimality
     of $F$ to specify some structural
     properties of $F$. After doing so,
     the proof of this alternative version of
     Lemma \ref{lemma:poly} is given by 
     the same argument as in the proof
     of such lemma, only by composing
     the stereographic projection
     generating our representation
     of polygon circle graphs with
     a symmetry of $\mathbb{R}$ with
     respect to the point 0.

     We now present our main result
     with similar techniques as the one
     employed in the proof of
     Theorem \ref{thm:circle}.

     \begin{thm}
         Let $F$ be a family of
         polygons with $\omega\left(F\right) = 2$.
         Then, $\chi\left(F\right) \leq 5$.
     \end{thm}
     \begin{proof}
         Let $F = \left\{h_{i}\right\}_{i = 1}^{n}$
         be a family of polygons with
         $\omega\left(F\right) = 2$.
         We construct the desired
         coloring inductively on $k$.

         For $k = 1$, let $F_1 \subseteq F$
         be the subset of polygons
         $h_{i} \in F$ such that 
         there is no $\left(a, b\right)
         \in P^{0}\left(F\right)$
         with $a < a_1^{\left(i\right)} <
         a_{k_{i}}^{\left(i\right)} < b$.
         By Lemma \ref{lemma:poly} there
         exists a coloring $f_1$ of $F_1$
         with colors $\left\{1, 2, 3\right\}$ 
         such that for every $h_{i} \in F_1$ 
         and $1 < j \leq k_{i}$,
         $\widetilde{F}^{0}\left(a_{j}^{\left(i\right)}\right)$ 
         have the same color.
         By the definition of $F_1$,
         each interval in $F \setminus F_1$ 
         is contained into some (unique)
         intersection $p \in P^{0}\left(F_1\right)$.
         Consider that the following construction 
         has been done for $k - 1$
         with $k \geq 2$. We want the 
         following properties to be true.
         \begin{enumerate}
             \item The polygons in $F' \defeq 
             \bigcup_{i =1}^{k-1} F_{i}$ 
             (where the $F_{i}$s might not
             be disjoint) are colored with colors
             $\left\{1, 2, 3, 4, 5\right\}$.
             \item The polygons in $F \setminus F'$ 
             do not overlap with
             intervals in
             $\bigcup_{i =1}^{k-2} F_{i} \setminus F_{k-1}$.
             \item All polygons
             $h_{i} \in F \setminus F'$ 
             are such that
             $a < a_1^{\left(i\right)}
             < a_{k_{i}}^{\left(i\right)} < b$
             for some
             $\left(a, b\right) \in P^{0}\left(F_{k-1}\right)$.
             \item For each
             $p \in \left(a^{\left(i', i\right)},
             b^{\left(i, i'\right)}\right)
             \in P^{0}\left(F_{k-1}\right)$ 
             either only one color is 
             used to color all polygons in
             $\overline{F}_{k-1}^{0}\left(a^{\left(i', i\right)}\right)$ 
             and no more than two colors are used
             to color intervals in
             $\widetilde{F}_{k-1}^{0}\left(b^{\left(i,i'\right)}\right)$.
         \end{enumerate}
         Let us show how to carry out
         the $k$-th step of this
         construction.
         If $F \setminus F' = \emptyset$,
         then we are done and we have
         the desired coloring.
         Consider any interval
         $p = \left(a^{\left(i', i\right)},
         b^{\left(i, i'\right)}\right)
         \in P^{0}\left(F_{k-1}\right)$,
         such that there
         exists $h_{j}$ with
         $a^{\left(i', i\right)} <
         a_1^{\left(j\right)} < a_{k_{j}}^{\left(j\right)}
         < b^{\left(i, i'\right)}$.
         Denote:
         \begin{align*}
             \overline{F}^{0}\left(a^{\left(i', i\right)}\right) &= 
             \left\{u_{j}\right\}_{j = 1}^{t} = 
             \left\{\left(c_1^{\left(j\right)}, \ldots, 
             c_{l_{j}}^{\left(j\right)}\right)\right\}_{j =1}^{t} \\
             \widetilde{F}^{0}\left(b^{\left(i, i'\right)}\right) &= 
             \left\{u_{j}\right\}_{j = t+1}^{s} = 
             \left\{\left(c_1^{\left(j\right)}, \ldots,
             c_{l_{j}}^{\left(j\right)}\right)\right\}_{j =t +1}^{s}.
         \end{align*}
         We note that the polygons
         $\left\{u_{j}\right\}_{j=1}^{t}$ 
         and $\left\{u_{j}\right\}_{j=t+1}^{s}$ 
         are ordered from innermost
         to outermost so that
         $p = \left(c_1^{s}, c_{l_{s}}^{\left(s\right)}\right)
         \cap \left(c_1^{\left(t\right)},
         c_{l_{t}}^{\left(t\right)}\right)$.
         Without loss of generality,
         suppose that $\gamma_1 \in 
         \left\{1, 3\right\}$ is the
         color of the polygons 
         $u_{t+1}, \ldots, u_{s}$ 
         and let $\gamma_2, \gamma_3$
         be the colors of the polygons
         $u_{t+1}, \ldots, u_{s}$ 
         and let $\gamma_2$, $\gamma_3$
         be the colors of the polygons
         $u_1, \ldots, u_{t}$.
         
         Let $I_{p}$ be the set of polygons
         where vertices are all
         in $p$, the polygons
         $u_{t+1}, \ldots, u_{s}$ and 
         the segment $\left(b^{\left(i, i'\right)},
         c_{l_{s}}^{\left(s\right)} + 1\right)$.
         Let $F_{k, p}$ be the set of
         polygons in $I_{p}$ whose 
         vertices are not all in
         the overlap of two polygons
         in $I_{p}$.
         By the same argument as the
         one used in the proof of 
         Theorem \ref{thm:circle},
         polygons in $I_{p} \setminus F_{k, p}$ 
         do not overlap with 
         any of the polygons of
         $F' \setminus \left\{u_{t+1}, \ldots
         , u_{s}\right\}$.
         If $p$ contains at least
         one interval, 
         then 
         $F_{k, p} \setminus 
         \left\{u_{t + 1}, \ldots,
         u_{s}\right\} \cup 
         \left\{b^{\left(i, i'\right)}, 
         c_{l_{s}}^{\left(s\right)}\right\}
         \neq \emptyset$.
         By Lemma \ref{lemma:poly},
         we can color $F_{k, p}$ with
         colors from 
         $\left\{1, 2, 3, 4, 5\right\} \setminus 
         \left\{\gamma_2, \gamma_3\right\}$
         such that for each $h_{i} \in F_{k, p}$,
         for all
         $1 \leq j < k_{i}$, the polygons in
         $\overline{F}_{k, p}^{0}\left(a_{j}^{\left(i\right)}\right)$ 
         share the same color. At the same time,
         since $\left\{u_{t +1}, \ldots, u_{s}\right\} = 
         \overline{F}_{k, p}^{0}\left(c_{l_{s}}^{\left(s\right)} + 1\right)$,
         we can assume that $u_{t + 1}, \ldots, u_{s}$ 
         is colored (only) with $\gamma_1$.

         Let us denote $F'_{k, p} \defeq
         F_{k, p} \setminus \left\{\left(
         b^{\left(i, i'\right)},
         c_{l_{s}}^{\left(s\right)} + 1\right)\right\}$.
         It is easy to see that the coloring of
         $F'_{k, p}$ is compatible with that 
         of $F'$. Carrying out similar 
         constructions for each 
         $p \in P^{0}\left(F_{k-1}\right)$ 
         containing at least one uncolored
         polygons of $F$, let 
         $F_{k} \defeq \bigcup_{p \in P^{0}\left(F_{k -1}\right)}
         F'_{k, p}$.

         One can ckeck that the induction
         hypotheses hold for $k$.
         Also, the number of uncolored polygons
         strictly decreases at each step.
         Therefore, the desired coloring will be completed
         in a finite number of iterations.
     \end{proof}    
\end{document}
    

