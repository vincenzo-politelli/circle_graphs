\documentclass[12pt]{article}

% format
% \usepackage[a4paper, total={6in, 9in}]{geometry}                         

% math symbols
\usepackage{amsmath}
\usepackage{amsthm}
\usepackage{amssymb}
\usepackage[shortlabels]{enumitem} 
\usepackage{mathtools}
\usepackage{bbm}

% annotations
\usepackage{todonotes}
\usepackage{pdfcomment}

% figures
\usepackage{graphicx}
\usepackage{subcaption}

% theorems
\theoremstyle{definition}
\newtheorem{thm}{Theorem}
\newtheorem{prop}[thm]{Proposition}
\newtheorem{lemma}[thm]{Lemma}
\newtheorem{rmk}[thm]{Remark}
\newtheorem{defn}[thm]{Definition}
\newtheorem{cor}[thm]{Corollary}
\newtheorem{exo}[thm]{Exercise}

% definition equal
\newcommand{\defeq}{\vcentcolon=}
\newcommand{\eqdef}{=\vcentcolon}
\pdfcommentsetup{color=yellow, opacity=0.5}

% bibliography
\usepackage[backend=biber,style=alphabetic, sorting=ynt]{biblatex}

% url highlight
\usepackage{hyperref}

% algos
\usepackage[linesnumbered,ruled,vlined]{algorithm2e}
\SetArgSty{textnormal}

% bullet point style
\renewcommand{\labelitemi}{\tiny$\blacksquare$}

% new commands
\DeclareMathOperator{\dom}{dom}
\DeclareMathOperator{\Hess}{\textbf{H}}
\DeclareMathOperator{\Diag}{Diag}
\DeclareMathOperator{\Tr}{Tr}
\DeclareMathOperator{\ind}{i}
\DeclareMathOperator{\sgn}{sign}

\begin{document}

    \section{Algorithms for coloring circle graphs}
    
    \textbf{WARNING}: intervals
    $h_1$, $h_2$ overlap if
    $h_1 \cap h_2 \neq \emptyset$
    $h_1 \not \subseteq h_2$ and
    $h_2 \not \subseteq h_1$.
    \vspace{4pt}

    We look at the class
    of circle graphs as the class
    of overlap graphs of intervals on a line.
    Without loss of generality,
    \pdfmarkupcomment{we only consider the interval
    models containing open intervals}
    {I am assumeng opennes}
    in which no two intervals
    share an endpoint.

    For any such family of intervals
    $F$ and any point $c$ on a line,
    we set $F^{-}\left(c\right) \defeq
    \left\{\left(a, b\right) \in F
    \middle| b < c\right\}$,
    $F^{0}\left(c\right) \defeq
    \left\{\left(a, b\right)
    \middle| a < c < b\right\}$,
    $F^{+}\left(c\right) \defeq
    \left\{\left(a, b\right)
    \middle| c < a\right\}$.
    If $\omega\left(F\right) = 2$,
    for all $\left(a, b\right) \in F$,
    we have that
    $F^{0}\left(a\right) \setminus F^{0}\left(b\right)$ 
    and $F^{0}\left(b\right) \setminus F^{0}\left(a\right)$
    only contain nested intervals.
    Otherwise, if we had two
    overlapping intervals $h_1, h_2 \in
    F^{0}\left(a\right) \setminus F^{0}\left(b\right)$,
    we would have $h_1, h_2, \left(a, b\right)$ 
    forming a triangle.

    \begin{lemma} \label{lemma:4.4}
        Let $F$ be the interval
        model of a circle graph.
        Suppose that $\omega\left(F\right) = 2$ 
        and that for no two
        $h_1, h_2 \in F$,
        there exists $h_3 \in F$ 
        with $h_3 \in h_1 \cap h_2$.
        Then, there exists a 3-coloring
        of $F$ such that 
        \pdfmarkupcomment{for all
        $\left(a, b\right) \in F$, 
        the intervals in $F^{0}\left(a\right) 
        \setminus F^{0}\left(b\right)$
        have the same color.}
        {I guess that the remarkable
        result is the 3-coloring
        part}
    \end{lemma}
    \begin{proof}
        By way of contradiction,
        suppose that the statement is incorrect
        and let $F = \left\{h_{i}\right\}_{i = 1}^{n} =
        \left\{\left(a_{i}, b_{i}\right)\right\}_{i = 1}^{n}$
        be the counterexample
        to this lemma with the
        least cardinality.
        Clearly, \pdfmarkupcomment{$G\left(F\right)$ is 
        connected.}{if G(F) was
        not connected, at least one
        of its connected components
        would countervene to the lemma
        and we would have a smaller
        counterexample}
        Let $\left(a_1, b_1\right),
        \ldots, \left(a_{t}, b_{t}\right)$,
        with $a_1 < \ldots < a_{t}$,
        be the intervals of $F$ which
        are not contained in any other 
        intervals. Notice that we have
        $F^{0}\left(a_{1}\right) = 
        F^{-}\left(a_1\right) = \emptyset$.
        Suppose that $\left|F^{0}\left(b_1\right)\right| \leq 1$.
        By the minimality of $F$,
        we get that the interval
        model $F' \defeq F \setminus \left\{h_{1}\right\}$ 
        is 3-colorable. It is clear
        that coloring $h_1$ with a different
        color from $h_2$, we obtain a coloring
        for $F$. Which contradicts the fact
        that $F$ is a counterexample to this lemma.
        Therefore, we conclude that $\left|F^{0}\left(b_1\right)\right| \geq 2$ 
        and $t \geq 2$.

        By the connectedness of $G\left(F\right)$,
        we have $h_{i} \cap h_{i+1} \neq \emptyset$ 
        for all $i \in \left\{1, \ldots, t -1\right\}$.
        Let us show that 
        \pdfmarkupcomment{for all
        $i \in \left\{1, \ldots, t - 1\right\}$
        we have
        $F^{-}\left(b_{i}\right) = \emptyset$ 
        or $F^{+}\left(a_{i+1}\right) = \emptyset$.}{That would imply
        that i = 1 or i = t 
        for all i in
        {1, \ldots, t}
        and so t <= 2}
        Suppose that this does not
        hold for some $1 \leq i \leq t - 1$.
        Let $F_1 = F \setminus F^{+}\left(a_{i+1}\right)$
        and $F_2 = F \setminus F^{-}\left(b_{i}\right)$.
        
        We claim that $F_1 \cap F_2 = 
        \left\{h_{i}, h_{i+1}\right\}$.
        Indeed, $h_{j} \in 
        F_1 \cap F_2$, 
        iff $a_{j} \leq a_{i+1}$ 
        and $b_{j} \geq b_{i+1}$.
        Clearly $h_{i}$ and
        $h_{i+1}$ have this
        property.
        Now, suppose that there
        exists $h_{j} \in F_1 \cap F_2$
        such that $a_{j} < a_{i+1}$
        and $b_{j} > b_{i+1}$.
        We cannot have $a_{j} < a_{i}$,
        because otherwise $h_{i} \subseteq h_{j}$.
        Similarly, we cannot have $b_{j} > b_{i+1}$,
        because otherwise, we would
        have $h_{i+1} \subseteq h_{j}$.
        Therefore we have $a_{i} < a_{j}$ 
        and $b_{j} < b_{i+1}$.
        The only remaining case is
        that of $a_{i} < a_{j} < a_{i+1}$ 
        and $b_{i} < b_{j} < b_{i+1}$.
        But if this is the case,
        $\left\{h_{i}, h_{j}, h_{i+1}\right\}$ would
        form a triangle, which
        contradicts our assumptions.
        
        In view of the minimality of
        $F$, there exist 3-colorings
        $f_1$, $f_2$ of $F_1$ and $F_2$ as required
        by the statement of this lemma
        with colors 1, 2 and 3 (here
        we use our assumption that
        $F^{-}\left(b_{i}\right)$ 
        and $F^{+}\left(a_{i+1}\right)$ 
        are non-empty and thus
        $F_{1}$ and $F_{2}$ are
        strictly smaller than $F$).
        Since $h_1$ and $h_2$ overlap,
        they have different colors
        both in $f_1$ and $f_2$, therefore,
        we can assume without loss of 
        generality that
        $f_1\left(h_{i}\right) = f_2\left(h_{i}\right) = 1$
        and $f_{1}\left(h_{i+1}\right) = 
        f_2\left(h_{i+1}\right) = 2$.
        Let
        \begin{equation*}
            f\left(h\right) = 
            \begin{cases}
                f_1\left(h\right) &\text{ if } h \in F_1, \\
                f_2\left(h\right) &\text{ if } h \in F_{2}.
            \end{cases}
        \end{equation*}
        We verify that $f$ 
        is a coloring of $F$. Let
        $h' = \left(a', b'\right) \in F$,
        $h'' = \left(a'', b''\right) \in F$
        with $a' < a'' < b' < b''$.
        If $\left\{h', h''\right\} \subseteq F_1$ 
        or $\left\{h', h''\right\} \subseteq F_2$,
        then $f\left(h'\right) \neq 
        f\left(h''\right)$.
        If $h' \in F^{-}\left(b_{i}\right)$ 
        and $h'' \in F^{+}\left(a_{i+1}\right)$
        then, by assumption
        we have
        $h', h'' \not \subseteq \left(a_{i+1}, b_{i}\right)$,
        and so we must have $a' < a_{i+1} < 
        a'' < b' < b_{i} < b''$. 
        Since $f_2$ is a coloring,
        we have $f\left(h''\right) =
        f_2\left(h''\right) =
        f_2\left(h_{i+1}\right) = 2$.
        Since $h'$ overlaps with $h_{i+1}$
        and $h', h_{i+1}
        \in F_1$, then $f\left(h'\right)
        \neq 2 = f\left(h''\right)$.
        \pdfcomment[icon=Note, color=yellow]
        {So far we proved that
        the colorings f1 and 
        f2 can be fused into 
        one coloring f.}
        Similarly, 
        \pdfmarkupcomment{we can verify
        that for all $\left(a, b\right) \in F$
        all intervals in $F\left(b\right)
        \setminus F\left(a\right)$
        have the same color.} 
        {We can do this by introducing 
        a trick like $\left(d, d_{s} + 1\right)$.
        This time it should be just
        $h_{i}$ added
        to $F_2$.}
        Thus
        $f$ is a 3-coloring of $F$,
        which contradicts our choice of $F$.
        \pdfmarkupcomment{
        Thus, we have that
        $F^{-}\left(b_{i}\right) = \emptyset$ and
        $F^{+}\left(a_{i+1}\right) = \emptyset$
        for all $i \in \left\{1, \ldots, t-1\right\}$.}
        {Now that we have this
        why do we go on?}

        We have $\left|F^{+}\left(a_{2}\right)\right| \geq
        \left|F^{0}\left(b_1\right)\right| \geq 2$,
        and, since $F^{-}\left(b_1\right) = \emptyset$,
        we have $F^{0}\left(a_2\right)
        = \left\{h_1\right\}$.
        
        If $F^{0}\left(b_2\right) = \emptyset$,
        i.e. $t=2$, then $h_2$
        only overlaps with $h_1$.
        A coloring $h_2$ with the color
        of the elements of $F^{0}\left(b_1\right)
        \setminus \left\{h_2\right\}$,
        we obtain a 3-coloring of $F$ with
        the conditions required by
        the present lemma, which is a
        contradiction.
        
        If $F^{0}\left(b_2\right) \neq \emptyset$
        and $t \geq 3$,
        we obtain (since $h_1 \in
        F^{-}\left(b_2\right)$)
        theat $F^{+}\left(a_3\right)=\emptyset$ 
        and thus $t=3$
        and $F^{0}\left(b_2\right) = \left\{h_3\right\}$.
        Consider $F \setminus h_2$.
        Since $F$ is minimal,
        we find that there exists
        a coloring $f$ of $F \setminus \left\{h_2\right\}$
        which follows the requirement
        of the present lemma.
        Let $h_2' = \left(a_2', b_2'\right)$
        be the longest interval of 
        $F^{0}\left(b_1\right) \setminus \left\{h_2\right\}$
        (which exists since $\left|F^{0}\left(b_1\right)\right|
        \geq 2$). This way
        $h_2'$ contains all
        the intervals in
        $F^{0}\left(b_1\right)$.
        Now, assume that $f\left(h_1\right) = 1$
        and $f\left(h_2'\right) = 2$.
        If $f\left(h_3\right)\neq 2$,
        then we have a coloring 
        of $F$ with the properties
        required by the present lemma.
        Which yields a contradiction.
        If $f\left(h_3\right) = 2$,
        then $h_3$ and $h_2'$
        do not overlap. Thus $h_3 \in F^{+}\left(b_2'\right)$.
        By assumption, all intervals
        of $F^{0}\left(b_2'\right)$ 
        are colored with the same color
        $\gamma \in \left\{1,3\right\}$.
        By recoloring all elements
        of $F^{+}\left(b_2'\right)$
        of color 2 to color 
        $\delta \in \left\{1,3\right\}
        \setminus \gamma$
        and all elements of $F^{+}\left(b_2'\right)$ 
        of color $\delta$ with color 2,
        \pdfmarkupcomment{
        we obtain a 
        new coloring $f'$ of 
        $F \setminus \left\{h_2\right\}$.} 
        {We essentially flip 2 and delta,
        if there was no conflict
        before, there is no conflict
        now} 
        \pdfmarkupcomment{Since $h_3 \in F^{+}\left(b_2'\right)$}
        {since h2' and h3 do not
        overlap}, we get
        $f'\left(h_2'\right)=2$ and
        $f'\left(h_3\right) \neq 2$.
        By coloring $h_2$ with
        color 2, 
        \pdfmarkupcomment{we obtain a coloring
        of $F$ with the properties
        required by the present lemma.}
        {this is because F0(a2)
        = {h1} and
        F0(b2) = {h3}}
    \end{proof}
    
    For each pair of overlapping
    intervals $h_1, h_2 \in F$,
    let $p\left(h_1, h_2\right)
    = h_1 \cap h_2$.
    We denote by $P\left(F\right)$ 
    the family of such intersections.
    Let $P^{0}\left(F\right) \subseteq 
    P\left(F\right)$ be the
    inclusion-wise maximal family
    of $P\left(F\right)$.

    \begin{lemma} \label{lemma:4.5}
        Let $F$ be a family of
        intervals with $\omega\left(F\right)=2$.
        Then, the intervals of $P^{0}\left(F\right)$ 
        do not intersect.
    \end{lemma}
    \begin{proof}
        \pdfmarkupcomment{
        Let $p_1=h_1 \cap h_2 \in 
        P^{0}\left(F\right)$ and
        $p_2=h_3 \cap h_4 \in P^{0}\left(F\right)$.}
        {Possibly h3=h2}
        Therefore, $\left(p_1 \cap p_2\right) =
        \left(a_4, b_1\right)$.
        So $a_1 < a_2 < a_4
        < b_1 <b_3 < b_4$.
        If $b_2 < b_4$,
        then $h_1, h_2, h_4$ 
        pairwise overlap. 
        Thus we have a triangle,
        which is a contradiction.
        Thus $b_4 < b_2$.
        By a symmetric argument
        we get $a_1 < a_3$.
        But then $p_2 =
        \left(a_4, b_3\right)
        \subsetneq \left(a_2, b_1\right)
        = p_1$ which contradicts
        the maximality of $p_2$.
    \end{proof}
    \begin{thm}
        Let $F$ be a family of intervals
        with $\omega\left(F\right)=2$.
        $\chi\left(F\right) \leq 5$.         
    \end{thm}        
    \begin{proof}
        \pdfmarkupcomment{
        We construct the desired
        coloring by induction
        on $k$.}
        {k is the 
        number of ``layers''}
        Let $k = 1$.
        Let $F_1$ be the subset
        of intervals of $F$ that 
        do not lie in the intersection
        of overlapping intervals of $F$.
        By Lemma \ref{lemma:4.4},
        there exists a coloring
        $f_1$ of $F_1$ with colors
        $\left\{1,2,3\right\}$ such
        that for every $\left(a,b\right)
        \in F_1$, the intervals
        of the family $F^{0}_{1}
        \left(b\right) \setminus F^{0}\left(a\right)$
        are colored with the same
        color.
        By Lemma \ref{lemma:4.5},
        the intervals of the family
        $P^{0}\left(F_1\right)$ 
        do not overlap. By definition
        of $F_1$, each interval in 
        $F \setminus F_1$ is contained
        in some (maximal) intersection
        $p \in P^{0}\left(F_1\right)$.
        Consider that the following
        construction has been done
        for $k - 1$ with $k \geq 2$.
        We want the following properties
        to be true.
        \begin{enumerate}
            \item The intervals in
                $F' \defeq 
                \bigcup_{i=1}^{k-1} F_{i}$
                \pdfmarkupcomment{
                (where the $F_{i}$s
                might not be disjoint)}
                {Aren't they?
                No. As we will see later
                we will have the intervals
                ht+1, ..., hs
                which are in F' 
                but also in Fp,k.}
                are colored with colors
                $\left\{1, \ldots, 5\right\}$.
            \item \pdfmarkupcomment{
                The intervals in
                $F \setminus F'$ do not
                overlap with intervals
                in $\bigcup_{i=1}^{k-2}
                F_{i} \setminus F_{k-1}$.}
                {Otherwise omega(F)
                >= 3}
                \label{cond:2}
            \item \label{cond:3} All intervals in 
                $F \setminus F'$ 
                \pdfmarkupcomment{
                are contained in some 
                interval $p \in 
                P^{0}\left(F^{k-1}\right)$.}
                {For k = 1, this is
                Lemma 2}
            \item For each $p =
                \left(c, d\right) \in 
                P^{0}\left(F_{k-1}\right)$ 
                either only one
                color is used to color
                all intervals in 
                $F^{0}_{k-1}\left(c\right) 
                \setminus F^{0}_{k-1}\left(d\right)$,
                and no more than 
                two colors are used to 
                colors intervals
                of $F^{0}_{k-1}\left(d\right)
                \setminus F^{0}_{k-1}\left(d\right)$
                or viceversa.
        \end{enumerate}
        Let us show how to carry out the 
        $k$-th step of this construction.
        If $F \setminus F' = \emptyset$,
        then we are done and we have 
        the desired coloring.
        Consider any interval
        $p = \left(c, d\right)
        \in P^{0}\left(F_{k-1}\right)$,
        \pdfmarkupcomment{
        which contains at least one interval
        of $F \setminus F'$.} 
        {By condition 3}
        Let
        \begin{align*}
            F^{0}_{k-1}\left(c\right) \setminus 
            F^{0}_{k-1}\left(d\right)
            &= \left\{h_{j}\right\}_{j=1}^{t}
            = \left\{\left(c_{j},
            d_{j}\right)\right\}_{j=1}^{t} \\
            F^{0}_{k-1}\left(d\right) \setminus 
            F^{0}_{k-1} \left(c\right)
            &= \left\{h_{j}\right\}_{j=t+1}^{s}
            = \left\{\left(c_{j},
            d_{j}\right)\right\}_{j=t+1}^{s},
        \end{align*}
        and $p = \left(c_{s}, d_{s}\right)
        \cap \left(c_{t}, d_{t}\right) = 
        \left(c_{s}, d_{t}\right)$.
        As already noted, 
        \pdfmarkupcomment{each
        family consists of
        nested intervals.}
        {Otherwise omega(F) >= 3}
        Without loss of generality,
        suppose that
        $\gamma_1 \in \left\{1, 3\right\}$
        is the color used to
        color the intervals
        $h_{t+1}, \ldots, h_{s}$
        and 
        \pdfmarkupcomment{let $\gamma_{2}, \gamma_{3}$}
        {Could also be only gamma2}
        be the colors used to 
        color
        $\left\{h_1, \ldots h_t\right\}$.
        
        Let $I_{p}$ be the
        set of intervals in $F$ 
        contained in $p$,
        the intervals $h_{t+1},
        \ldots, h_{s}$ and the 
        interval 
        $\left(d, d_{s} + 1\right)$.
        Let $F_{k, p}$ be 
        the set of intervals in $I_{p}$ 
        which are not contained in
        the overlap of any two
        intervals in $I_{p}$. 

        Let us show that the 
        intervals from $I_{p}
        \setminus F_{k, p}$ 
        \pdfmarkupcomment{do 
        not overlap with 
        any of the intervals of
        $F' \setminus \left\{
        h_{t+1},\ldots,h_{s}\right\}$.}
        {The following
        levels can only 
        overlap with 
        {ht+1,
        ..., hs}}
        By condition \ref{cond:2}
        of our induction hypothesis, 
        intervals from
        $F' \setminus \left\{h_{t+1},
        \ldots, h_{s}\right\}$
        that can overlap with 
        elements in $I_{p} \setminus 
        F_{k,p}$ 
        \pdfmarkupcomment{are only intervals
        in $\left\{h_1, \ldots, h_{t}\right\}$.}
        {This is true for intervals
        overlapping with p, but what
        about intervals overlapping
        with d?
        The only intervals in
        Ip overlapping
        with (d, ds + 1) 
        are ht+1,...,
        hs which are
        also in Fk,p.}
        Let $\left(a_1, b_1\right) \in I_{p}$ 
        be an interval contained in the
        overlapping of
        $\left(a_2, b_2\right), 
        \left(a_3, b_3\right)
        \in I_{p}$ and thus
        overlapping with some
        $h_{j} =\left(c_{j}, d_{j}\right)$ 
        for $1 \leq j \leq t-1$.
        Then $a_{i} > c_{s}$ for
        all $i \in \left\{1, 2, 3\right\}$.
        We also get $c_{j} < c_{s}$,
        $a_1 < d_{j} < b_1$. Hence
        $h_{j}$, $\left(a_2, b_2\right)$ 
        and $\left(a_3, b_3\right)$ 
        pairwise overlap.
        Which is a contradiction.

        If $p$ contains at least one interval,
        then $F_{k, p} \setminus \left(h_{t+1},
        \ldots, h_{s}\right) \cup 
        \left\{\left(d, d_{s} + 1\right)\right\}
        \neq \emptyset$.
        By Lemma \ref{lemma:4.4}
        (replacing $F^{0}\left(b\right)
        \setminus F^{0}\left(a\right)$
        with $F^{0}\left(a\right)
        \setminus F^{0}\left(b\right)$),
        we can color $F_{k, p}$ 
        with colors from
        $\left\{1,2,3,4,5\right\} \setminus 
        \left\{\gamma_2, \gamma_3\right\}$
        such that
        for each
        $\left(a, b\right) \in 
        F_{k, p}$,
        the interval from
        $F^{0}_{k, p}\left(a\right) \setminus 
        F^{0}_{k, p}\left(b\right)$
        have the same color.
        At the same time,
        \pdfmarkupcomment{since $\left\{h_{t+1},
        \ldots, h_{s}\right\}
        = F^{0}_{k, p}\left(d\right)
        \setminus F^{0}\left(d_{s} + 1\right)$}
        {Actually 
        F0(ds + 1) 
        is empty, but we want
        to apply the statement
        of Lemma 1
        rigorously. This
        is why we introduce
        the segment (d, ds + 1) 
        in the first place.}
        we can assume that 
        $h_{t+1}, \ldots,
        h_{s}$ is colored 
        with (only) $\gamma_1$.
        
        Let us denote $F_{k, p}' = 
        F_{k, p} \setminus \left\{
        \left(d, d_{s} + 1\right)\right\}$.
        It is easy to see that the coloring of
        $F_{k, p}'$ is compatible
        with that of
        $F'$.
        Carrying out similar constructions
        for each $p \in P^{0}\left(F_{k-1}\right)$
        containing at least one uncolord
        interval of $F$, let
        $F_{k} \defeq \bigcup_{p \in 
        P^{0}\left(F_{k-1}\right)} F_{k, p}'$.

        One can check that the 
        induction hypotheses hold
        for $k$.
        Also, the number of uncolored
        vertices strictly decreseas 
        at each step. Therefore
        the coloring will be 
        completed in a finite
        number of iterations.
    \end{proof}
\end{document}
    

